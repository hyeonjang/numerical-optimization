\documentclass[12pt,letterpaper]{article}
\usepackage{fullpage}
\usepackage[top=2cm, bottom=4.5cm, left=2.5cm, right=2.5cm]{geometry}
\usepackage{amsmath,amsthm,amsfonts,amssymb,amscd}
\usepackage{lastpage}
\usepackage{enumerate}
\usepackage{fancyhdr}
\usepackage{mathrsfs}
\usepackage{xcolor}
\usepackage{graphicx}
\usepackage{listings}
\usepackage{hyperref}
\usepackage[section]{minted}
\usepackage{hyperref}
\usepackage{multirow}
\definecolor{mintedbackground}{rgb}{0.95,0.95,0.95}

\hypersetup{%
  colorlinks=true,
  linkcolor=blue,
  linkbordercolor={0 0 1}
}
 
\renewcommand\lstlistingname{method}
\renewcommand\lstlistlistingname{Algorithms}
\def\lstlistingautorefname{Alg.}

\colorlet{mygreen}{green!60!blue}

\newmintedfile[cppcode]{cpp}{
bgcolor=mintedbackground,
fontfamily=tt,
linenos=true,
numberblanklines=true,
numbersep=5pt,
gobble=0,
frame=leftline,
framerule=0.4pt,
framesep=2mm,
funcnamehighlighting=true,
tabsize=4,
obeytabs=false,
mathescape=false
samepage=true, %with this setting you can force the list to appear on the same page
showspaces=false,
showtabs =false,
texcl=false,
fontsize=\small
}

\setlength{\parindent}{0.0in}
\setlength{\parskip}{0.05in}

% Edit these as appropriate
\newcommand\course{EC6301}
\newcommand\name{Numerical Opimization}
\newcommand\hwnumber{2}                  % <-- homework number
\newcommand\NetIDa{20211046}           % <-- NetID of person #1
\newcommand\NetIDb{Hyeonjang An}           % <-- NetID of person #2 (Comment this line out for problem sets)
\newcommand\github{\url{https://github.com/hyeonjang/numerical-optimization}}

\pagestyle{fancyplain}
\headheight 35pt
\lhead{\github\\\NetIDa\\\NetIDb}                 % <-- Comment this line out for problem sets (make sure you are person #1)
\chead{\textbf{\Large Homework \hwnumber}}
\rhead{\course \\ \name \\ \today}
\lfoot{}
\cfoot{}
\rfoot{\small\thepage}
\headsep 1.5em

\begin{document}

\section*{Problem}

Discuss comparative study in terms of convergence speed between search algorithms for at least four optimization problems you generated accordingly.

\begin{enumerate}
\item Target functions
\begin{itemize}
\item From function1 to function4, they are same as assignment1. 
function1 is general quaratic function. function2 is log and function3 is trigonometric function. 
function4 is the minus signed version of gaussian function. 
function5 and function6 are newly added for testing nondifferentiable cases.
\end{itemize}
\begin{center}
    \begin{tabular}{| c | l | c | c | } \hline
                  & \multirow{2}{*}{functions}            & \multicolumn{2}{c|}{performance} \\ \cline{3-4}
                  &                                       & fibonacci search & golden section search  \\ \hline
        function1 & $f(x)=x^4 +2x^3-3x^2-10x+7$           &  14643ns         & 14453ns   \\
        function2 & $f(x)=x\ln(x)$                        &  6123ns          & 5552ns    \\
        function3 & $f(x)=\sin(x)+x^2-10$                 &  5604ns          & 5120ns    \\
        function4 & $f(x)=-\exp(-\frac{x^2}{\sigma ^2})$  &  5569ns          & 5023ns    \\
        function5 & $f(x)=|x-0.3|$                        &  5020ns          & 4075ns    \\
        function6 & $f(x)=|\ln(x)|$                       &  5011ns          & 4496ns    \\
        \hline
    \end{tabular}
\end{center}

\item Conditions
\begin{itemize}
\item The bound is determined by the seeking bound algorithm. The initial random values to search bound are chosen by random\_int function.
\end{itemize}

\item Analysis
\begin{itemize}
\item The maximum iteration is set by 46, because of the limitation for maximum fibonacci sequence value.
The maximum integer value is now 2,147,483,647, but the 47th fibonacci value is 2,971,215,073. 
If more complicated Implementation is added, the fibonacci sequence could be larger. But currently didn't.
Therefore, the maximum iteration is limited as 46, and the performance is related to this.
\item 

\end{itemize}
\end{enumerate}

\newpage
\section*{Implementation}

\begin{enumerate}

\item Class: Optimizing Method

\cppcode
[
  firstline=9, lastline=47
]
{../../code/method.h}

\item Seeking bound
\begin{itemize}
  \item 
\end{itemize}
\cppcode
[ 
  firstline=201, lastline=251
]
{../../code/method.cpp}

\item Fibonacci search
\begin{itemize}
  \item Construction of Fibonacci
  \item Due to the maximum integer value is limited by 214748364, the maximum index of fibonacci sequence is currently 46. If in the case of unsigned or long integer, it could be changed
\end{itemize}
\cppcode
[
  firstline=187, lastline=199
]
{../../code/method.cpp}
\begin{itemize}
  \item Fibonacci search
\end{itemize}
\cppcode
[
  firstline=107, lastline=151,
  % breaklines
]
{../../code/method.cpp}

\newpage

\item Golden section search
\cppcode
[
  firstline=153, lastline=186
]
{../../code/method.cpp}

\end{enumerate}

\end{document}