\documentclass[12pt,letterpaper]{article}
\usepackage{fullpage}
\usepackage[top=2cm, bottom=4.5cm, left=2.5cm, right=2.5cm]{geometry}
\usepackage{amsmath,amsthm,amsfonts,amssymb,amscd}
\usepackage{lastpage}
\usepackage{enumerate}
\usepackage{fancyhdr}
\usepackage{mathrsfs}
\usepackage{xcolor}
\usepackage{graphicx}
\usepackage{listings}
\usepackage{hyperref}
\usepackage{minted}

\hypersetup{%
  colorlinks=true,
  linkcolor=blue,
  linkbordercolor={0 0 1}
}
 
\renewcommand\lstlistingname{method}
\renewcommand\lstlistlistingname{Algorithms}
\def\lstlistingautorefname{Alg.}

\colorlet{mygreen}{green!60!blue}

\lstdefinestyle{C++}{
    language        = C++,
    frame           = single,
    backgroundcolor = \color{gray!10}, 
    basicstyle      = \ttfamily,
    columns         = fullflexible,
    breaklines      = true,
    keywordstyle    = \color{blue},
    stringstyle     = \color{green},
    commentstyle    = \color{mygreen}\ttfamily
}

\setlength{\parindent}{0.0in}
\setlength{\parskip}{0.05in}

% Edit these as appropriate
\newcommand\course{EC6301}
\newcommand\name{Numerical Opimization}
\newcommand\hwnumber{2}                  % <-- homework number
\newcommand\NetIDa{20211046}           % <-- NetID of person #1
\newcommand\NetIDb{Hyeonjang An}           % <-- NetID of person #2 (Comment this line out for problem sets)

\pagestyle{fancyplain}
\headheight 35pt
\lhead{\NetIDa}
\lhead{\NetIDa\\\NetIDb}                 % <-- Comment this line out for problem sets (make sure you are person #1)
\chead{\textbf{\Large Homework \hwnumber}}
\rhead{\course \\ \name \\ \today}
\lfoot{}
\cfoot{}
\rfoot{\small\thepage}
\headsep 1.5em

\begin{document}

\section*{Problem}

Discuss thier comparative performance for at least four different problems you generate.

\begin{enumerate}
\item Target functions and derivative
\\ Function1 is general quaratic function, which has derivative as cubic. function2 is log and function3 is trigonometric function. 
function4 is the minus signed version of gaussian function, which is usally used as kernel. It's $\sigma$ value is set as 1.4
\begin{center}
    \begin{tabular}{| c | c | c | c |}
        \hline
        function  & original function                   & derivation of function           & interval \\
        \hline
        function1 & $f(x)=x^4 +2x^3-3x^2-10x+7$         & $f'(x) = 4x^3 + 6x^2 - 6x - 10$  & [-5, 5]  \\
        function2 & $f(x)=x\ln(x)$                      & $f'(x) = \ln(x) + 1 $            & [0.1, 5]  \\
        function3 & $f(x)=\sin(x)+x^2-10$               & $f'(x) = \cos(x)+2x$             & [-5, 5]  \\
        function4 & $f(x)=-\exp(-\frac{x^2}{\sigma ^2})$ & $f'(x) = \frac{x}{\sigma^2}\exp(-\frac{x^2}{2\sigma^2})$ & [-1, 1] \\
        \hline
    \end{tabular}
\end{center}

\item Conditions
\begin{itemize}
    \item Within the interval, all functions are continuous and the first order derivative of those are also continuous.

    \item In the case of bracketing method (bisection \& regular falsi) 
    \\ Within the interval the function has the value zero. This can be calculated analytically.

    \item In the case of straight line method (Newton's \& secant)
    \\ For the comparision pairness, the initial points of each methods are same as the maximum point of interval, which is used in bracketing method.

\end{itemize}

\item Peformance comparision
\begin{center}
\begin{tabular}{ | c | c | c | c | c |}
    \hline
     &          bisection & Newton's & secant & regular falsi \\
    \hline
    function1 &  2072ns     & 546ns     & 789ns  &  11241ns \\
    function2 &  2639ns     & 189ns     & 892ns  &  5778ns  \\
    function3 &  2428ns     & 376ns     & 400ns  &  1310ns  \\
    function4 &  105ns      & 213ns     & 36.9ns &  238ns   \\
    \hline
\end{tabular}
\end{center}

\item Analysis
\\ Apparently, the convergence of Newton's method is the fastest.
Moreover, the overhead of regular falsi method is bigger than I thought. 
In the case of function1, regular falsi method has the slowest convergence rate.
\\ The speical thing is function 4. In the case of gaussian function, Newton's method has the slowest.
I think that it is because of the calculation overhead from derivation.


\end{enumerate}

\newpage
\section*{Implementation}

Implement the method of bisection , Newtons's, secant, regular falsi.

\begin{enumerate}

\item Optimizing Method Class
\inputminted
  [frame=single, fontsize=\small]
  {cpp}
  {../../code/method.h}

\item Seeking bound
\inputminted
  [ 
    frame=single, fontsize=\small, 
    firstline=196, lastline=254
  ]
  {cpp}
  {../../code/method.cpp}

\item Fibonacci search
\inputminted
  [
    frame=single, fontsize=\small,
    firstline=118, lastline=140
  ]
  {cpp}
  {../../code/method.cpp}

\item Golden section search
\inputminted
  [
    frame=single, fontsize=\small, 
    firstline=147, lastline=173
  ]
  {cpp}
  {../../code/method.cpp}

\end{enumerate}

\end{document}