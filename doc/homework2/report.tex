\documentclass[12pt,letterpaper]{article}
\usepackage{fullpage}
\usepackage[top=2cm, bottom=4.5cm, left=2.5cm, right=2.5cm]{geometry}
\usepackage{amsmath,amsthm,amsfonts,amssymb,amscd}
\usepackage{lastpage}
\usepackage{enumerate}
\usepackage{fancyhdr}
\usepackage{mathrsfs}
\usepackage{xcolor}
\usepackage{graphicx}
\usepackage{listings}
\usepackage{hyperref}
\usepackage[section]{minted}
\usepackage{hyperref}
\definecolor{mintedbackground}{rgb}{0.95,0.95,0.95}

\hypersetup{%
  colorlinks=true,
  linkcolor=blue,
  linkbordercolor={0 0 1}
}
 
\renewcommand\lstlistingname{method}
\renewcommand\lstlistlistingname{Algorithms}
\def\lstlistingautorefname{Alg.}

\colorlet{mygreen}{green!60!blue}

\newmintedfile[cppcode]{cpp}{
bgcolor=mintedbackground,
fontfamily=tt,
linenos=true,
numberblanklines=true,
numbersep=5pt,
gobble=0,
frame=leftline,
framerule=0.4pt,
framesep=2mm,
funcnamehighlighting=true,
tabsize=4,
obeytabs=false,
mathescape=false
samepage=true, %with this setting you can force the list to appear on the same page
showspaces=false,
showtabs =false,
texcl=false,
fontsize=\small
}

\setlength{\parindent}{0.0in}
\setlength{\parskip}{0.05in}

% Edit these as appropriate
\newcommand\course{EC6301}
\newcommand\name{Numerical Opimization}
\newcommand\hwnumber{2}                  % <-- homework number
\newcommand\NetIDa{20211046}           % <-- NetID of person #1
\newcommand\NetIDb{Hyeonjang An}           % <-- NetID of person #2 (Comment this line out for problem sets)
\newcommand\github{\url{https://github.com/hyeonjang/numerical-optimization}}

\pagestyle{fancyplain}
\headheight 35pt
\lhead{\github\\\NetIDa\\\NetIDb}                 % <-- Comment this line out for problem sets (make sure you are person #1)
\chead{\textbf{\Large Homework \hwnumber}}
\rhead{\course \\ \name \\ \today}
\lfoot{}
\cfoot{}
\rfoot{\small\thepage}
\headsep 1.5em

\begin{document}

\section*{Problem}

Discuss comparative study in terms of convergence speed between search algorithms for at least four optimization problems you generated accordingly.

\begin{enumerate}
\item Target functions and bound
\\ Bounds are computed automatically by seeking bound algorithm.
\\ From function1 to function4, they are same as assignment1. The function5 and function6 are newly added for testing nondifferentiable cases.
\begin{center}
    \begin{tabular}{| c | l | }
        \hline
                  & functions                             \\
        \hline
        function1 & $f(x)=x^4 +2x^3-3x^2-10x+7$           \\
        function2 & $f(x)=x\ln(x)$                        \\
        function3 & $f(x)=\sin(x)+x^2-10$                 \\
        function4 & $f(x)=-\exp(-\frac{x^2}{\sigma ^2})$  \\
        function5 & $f(x)=|x-0.3|$                        \\
        function6 & $f(x)=|\ln(x)|$                       \\
        \hline
    \end{tabular}
\end{center}

\item Peformance comparision
\begin{center}
\begin{tabular}{ | c | c | c |}
    \hline
              & fibonacci search & golden section search \\
    \hline
    function1 &  14643ns    & 14453ns   \\
    function2 &  6123ns     & 5552ns    \\
    function3 &  5604ns     & 5120ns    \\
    function4 &  5569ns     & 5023ns    \\
    function5 &  5020ns     & 4075ns    \\
    function5 &  5011ns     & 4496ns    \\
    \hline
\end{tabular}
\end{center}

\item Conditions
\\ As said before, the bound is determined by the seeking bound algorithm.

\item Analysis
\\ The maximum iteration is set by 46, because of the limitation for maximum fibonacci sequence value.
\end{enumerate}

\newpage
\section*{Implementation}

\begin{enumerate}

\item Class: Optimizing Method

\cppcode
[
  firstline=9, lastline=47
]
{../../code/method.h}

\item Seeking bound
\cppcode
[ 
  firstline=201, lastline=251
]
{../../code/method.cpp}

\item Fibonacci search
\begin{itemize}
  \item Construction of Fibonacci
  \item Due to the maximum integer value is limited by 214748364, the maximum index of fibonacci sequence is currently 46. If in the case of unsigned or long integer, it could be changed
\end{itemize}
\cppcode
[
  firstline=187, lastline=199
]
{../../code/method.cpp}
\begin{itemize}
  \item Fibonacci search
\end{itemize}
\cppcode
[
  firstline=107, lastline=151,
  % breaklines
]
{../../code/method.cpp}

\newpage

\item Golden section search
\cppcode
[
  firstline=153, lastline=185
]
{../../code/method.cpp}

\end{enumerate}

\end{document}