% !TEX root=../main.tex
\documentclass{standalone}

\begin{document}
  \begin{enumerate}

    % ##############################################################
    \item \textbf{Chromosome \& Population}
    \begin{enumerate}
      \item Due to the implementation convienience, \emph{chromosome\_t} and \emph{population\_t} are predefined. \emph{chromosome\_t} is a encoded binary type and \emph{population\_t} contains a \emph{chromosome\_t} and their fitness value.
      \item It is more complicate to convert floating point value to binary value than to convert integer value. I used a trick here. As shown in the code, there is no encoding function converting from floating point to binary, there is just a decoding function. The binary value of \emph{chromosome\_t} is automatically generated by bernoulli distribution. After then, when it is decoded, the chromosome value becomes normalized into range 0 to 1 by the value of a maximum bitstring.
    \end{enumerate}
    \begin{itemize}[label=\quad,leftmargin=-5em]
      \item \cppcode[]{../../code/global/ga_helper.h}
    \end{itemize}

    \newpage % ##############################################################
    \item \textbf{GeneticAlgorithm}
    \begin{enumerate}
      \item Initialization
      \begin{itemize}
        \item The popuations are initialized as given size by population\_t and their fitness values are also evaluated. And all this processes are done in the construction of \emph{GeneticAlgorithm}.
      \end{itemize}
      \item Selection
      \begin{itemize}
        \item I implemented roulette wheel selection as selection method
        \item Because we want to find minimum value of the given function, the probabilty to choose chromosome is computed reversely from the fitness values \\
        $ prob = (1 - fitness/fitness\_sum)/the number of chromosomes $
      \end{itemize}
      \item Reproduction
      \begin{itemize}
        \item As a reproduction, crossover and mutation can occur to the selected chromosomes within the given probability. The performance related to this will be discussed later.
        \item crossover - point slicing method using twopoint, it is determined by uniform distribution in the chromosome length
      \end{itemize}
      \item Replacement
      \begin{itemize}
        \item As the replacement policy, there are the two methods. 
        \item One of these is \emph{replace\_parent}, which replace the selected chromosomes to the new chromosomes that are crossovered or mutated.
        \item The other is \emph{replace\_worst}. By searching the worst fitness value within the populations, we can replace the chromosome which has the worst fitness value to a new chromosome generated by crossover or mutation method.
        \item In the performance comparison part, I fixed the replacement method as \emph{replace\_worst}
      \end{itemize}
    \end{enumerate}

    \begin{itemize}[label=\quad,leftmargin=-5em]
      \item \cppcode[]{../../code/global/genetic_algorithm.hpp}
    \end{itemize}
  % ##############################################################

  \end{enumerate}
\end{document}