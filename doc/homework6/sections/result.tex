% !TEX root=../main.tex
\documentclass{standalone}

\begin{document}
\begin{enumerate}
  \item Convergence
  \begin{itemize}
    \item When the descent condition is satisfied, which means the $the value of \lambda is small$, the two method behavior is almost same. 
    The first function case shows this phenomenon.
    \item Depending on the initial point, Gauss-Newton's method can be failed to converge. 
    In constrast to this, LM method shows more stable behavior to initial points.
    For example, as shown in the table, in the second function case, 
    gauss-newton's method faild to converge, when the initial points are [-2, -2, 2, 2] and [10, 10, 10, 10]
  \end{itemize}
\end{enumerate}
\begin{center}
  \begin{tabular}{| c | c | c | c } \hline
  \multirow{2}{*}{initial} & \multirow{2}{*}{$f(a,b,c,d;t)$}   & \multicolumn{2}{c|}{Final point$(a, b, c, d)$} \\ \cline{3-4}
                                  &                              & Gauss-Newton's                    & LM \\ \hline
  \multirow{2}{*}{[-1, -1, 1, 1]}   & (a)                        & 0.00171, 0.00071, -0.00352, 0.253 & 0.00171, 0.00071, -0.00352, 0.253  \\
                                    & (b)                        & 10412, -22924.2, 31402.8, 6338.3  & 5.33133, 5.68737, 5.60708, 9.92438 \\ \hline
  \multirow{2}{*}{[-2, -2, 2, 2]}   & (a)                        & 0.00171, 0.00071, -0.00352, 0.253 & 0.00171, 0.00071, -0.00352, 0.253  \\ 
                                    & (b)                        & 63.6425, 27.2601, -46.5018, 5.0931 & 5.33133, 5.68737, 5.60708, 9.92438 \\ \hline
  \multirow{2}{*}{[10,10,10,10]} & (a)                        & 0.00171, 0.00071, -0.00352, 0.253  & 0.00171, 0.00071, -0.00352, 0.253  \\ 
                                    & (b)                        & 5.33133, 5.68737, 5.60708, 9.92438 & 5.33133, 5.68737, 5.60708, 9.92438 \\ \hline
  \end{tabular}
\end{center}

\end{document}