\documentclass[12pt,letterpaper]{article}
\usepackage{fullpage}
\usepackage[top=2cm, bottom=4.5cm, left=2.5cm, right=2.5cm]{geometry}
\usepackage{amsmath,amsthm,amsfonts,amssymb,amscd}
\usepackage{lastpage}
\usepackage{enumerate}
\usepackage{fancyhdr}
\usepackage{mathrsfs}
\usepackage{xcolor}
\usepackage{graphicx}
\usepackage{listings}
\usepackage{hyperref}

\hypersetup{%
  colorlinks=true,
  linkcolor=blue,
  linkbordercolor={0 0 1}
}
 
\renewcommand\lstlistingname{method}
\renewcommand\lstlistlistingname{Algorithms}
\def\lstlistingautorefname{Alg.}

\colorlet{mygreen}{green!60!blue}

\lstdefinestyle{C++}{
    language        = C++,
    frame           = single,
    backgroundcolor = \color{gray!10}, 
    basicstyle      = \ttfamily,
    columns         = fullflexible,
    breaklines      = true,
    keywordstyle    = \color{blue},
    stringstyle     = \color{green},
    commentstyle    = \color{mygreen}\ttfamily
}

\setlength{\parindent}{0.0in}
\setlength{\parskip}{0.05in}

% Edit these as appropriate
\newcommand\course{EC6301}
\newcommand\name{Numerical Opimization}
\newcommand\hwnumber{1}                  % <-- homework number
\newcommand\NetIDa{20211046}           % <-- NetID of person #1
\newcommand\NetIDb{Hyeonjang An}           % <-- NetID of person #2 (Comment this line out for problem sets)

\pagestyle{fancyplain}
\headheight 35pt
\lhead{\NetIDa}
\lhead{\NetIDa\\\NetIDb}                 % <-- Comment this line out for problem sets (make sure you are person #1)
\chead{\textbf{\Large Homework \hwnumber}}
\rhead{\course \\ \name \\ \today}
\lfoot{}
\cfoot{}
\rfoot{\small\thepage}
\headsep 1.5em

\begin{document}

\section*{Problem}

Discuss thier comparative performance for at least four different problems you generate.

\begin{enumerate}
\item Target functions and derivative
\\ Function1 is general quaratic function, which has derivative as cubic. function2 is log and function3 is trigonometric function. 
function4 is the minus signed version of gaussian function, which is usally used as kernel. It's $\sigma$ value is set as 1.4
\begin{center}
    \begin{tabular}{| c | c | c | c |}
        \hline
        function  & original function                   & derivation of function           & interval \\
        \hline
        function1 & $f(x)=x^4 +2x^3-3x^2-10x+7$         & $f'(x) = 4x^3 + 6x^2 - 6x - 10$  & [-5, 5]  \\
        function2 & $f(x)=x\ln(x)$                      & $f'(x) = \ln(x) + 1 $            & [0.1, 5]  \\
        function3 & $f(x)=\sin(x)+x^2-10$               & $f'(x) = \cos(x)+2x$             & [-5, 5]  \\
        function4 & $f(x)=-\exp(-\frac{x^2}{\sigma ^2})$ & $f'(x) = \frac{x}{\sigma^2}\exp(-\frac{x^2}{2\sigma^2})$ & [-1, 1] \\
        \hline
    \end{tabular}
\end{center}

\item Conditions
\begin{itemize}
    \item Within the interval, all functions are continuous and the first order derivative of those are also continuous.

    \item In the case of bracketing method (bisection \& regular falsi) 
    \\ Within the interval the function has the value zero. This can be calculated analytically.

    \item In the case of straight line method (Newton's \& secant)
    \\ For the comparision pairness, the initial points of each methods are same as the maximum point of interval, which is used in bracketing method.

\end{itemize}

\item Peformance comparision
\begin{center}
\begin{tabular}{ | c | c | c | c | c |}
    \hline
     &          bisection & Newton's & secant & regular falsi \\
    \hline
    function1 &  2072ns     & 546ns     & 789ns  &  11241ns \\
    function2 &  2639ns     & 189ns     & 892ns  &  5778ns  \\
    function3 &  2428ns     & 376ns     & 400ns  &  1310ns  \\
    function4 &  105ns      & 213ns     & 36.9ns &  238ns   \\
    \hline
\end{tabular}
\end{center}

\item Analysis
\\ Apparently, the convergence of Newton's method is the fastest.
Moreover, the overhead of regular falsi method is bigger than I thought. 
In the case of function1, regular falsi method has the slowest convergence rate.
\\ The speical thing is function 4. In the case of gaussian function, Newton's method has the slowest.
I think that it is because of the calculation overhead from derivation.


\end{enumerate}


\newpage
\section*{Implementation}

Implement the method of bisection , Newtons's, secant, regular falsi.

\begin{enumerate}

\item Optimizing Method Class
\begin{lstlisting}[style=C++]
class Method {
public:
  Method(std::function<float(const float&)> f):function(f){};

  // optimization methods
  float bisection(float start, float end);
  float newtons(float x);
  float secant(float x1, float x0);
  float regular_falsi(float start, float end);

protected:
  // target function as member
  std::function<float(const float&)> function;
};
\end{lstlisting}

\item Bisection method
\lstset{caption={bisection}}
\begin{lstlisting}[style=C++]
float Method::bisection(float start, float end) {
  assert( function(start)*function(end)<0 );

  auto midpoint = (start + end)/2.f;

  if(function(midpoint)==0 || end-start<MIN)
    return midpoint;

  if(function(midpoint)*function(start)<0)
    midpoint = bisection(start, midpoint);
  else
    midpoint = bisection(midpoint, end);

  return midpoint;
}
\end{lstlisting}

\newpage

\item Newtons's method
\lstset{caption={Newton's}}
\begin{lstlisting}[style=C++]
float Method::newtons(float x0) {
  // approximattion of derivative lambda function
  auto d = 
  [](std::function<float(const float&)> func, float x, float eps=1e-6) 
  { 
      return (func(x+eps) - func(x))/eps;
  };

  float x1 = x0;
  while(function(x1)>0.f) {
      float t = x1;
      x1 = t - function(t)/d(function, t);
  }
  return x1;
}
\end{lstlisting}

\item Secant method
\lstset{caption={secant}}
\begin{lstlisting}[style=C++]
// Two point approximation method
float Method::secant(float x1, float x0){
    // no matter which one is bigger
    float t1 = std::min(x1, x0);
    float t0 = std::max(x1, x0);

    // initial two points
    float x2 = MAX;
    while(function(x2)>0.f)
    {
        x2 = t1 - ((t1-t0)/(function(t1)-function(t0))) * function(t1);

        t0 = t1;
        t1 = x2;
    }

    return x2;
}
\end{lstlisting}

\newpage

\item Regular-falsi method
\lstset{caption={regular false}}
\begin{lstlisting}[style=C++]
// recursive version
float Method::regular_falsi(float start, float end){
    // secant method lambda
    auto sec = [](std::function<float(const float&)> func, float x1, float x0)
    { 
        return x1 - ((x1-x0)/(func(x1)-func(x0))) * func(x1); 
    };

    assert( function(start)*function(end)<0 );

    // new x-axis intersection point
    float x = sec(function, start, end);

    if ( end-start<MIN )
        return x;
    
    // almost zero
    if( function(x)==0 || -MIN<function(x) && function(x)<MIN ) 
        return x;

    // do recursivly until the end
    if( function(start) * function(x) < 0)
        x = regular_falsi(start, x);
    else if ( function(end) * function(x) < 0)
        x = regular_falsi(x, end);

    return x;
}
\end{lstlisting}

\end{enumerate}

\end{document}