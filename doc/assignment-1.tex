\documentclass[12pt,letterpaper]{article}
\usepackage{fullpage}
\usepackage[top=2cm, bottom=4.5cm, left=2.5cm, right=2.5cm]{geometry}
\usepackage{amsmath,amsthm,amsfonts,amssymb,amscd}
\usepackage{lastpage}
\usepackage{enumerate}
\usepackage{fancyhdr}
\usepackage{mathrsfs}
\usepackage{xcolor}
\usepackage{graphicx}
\usepackage{listings}
\usepackage{hyperref}

\hypersetup{%
  colorlinks=true,
  linkcolor=blue,
  linkbordercolor={0 0 1}
}
 
\renewcommand\lstlistingname{method}
\renewcommand\lstlistlistingname{Algorithms}
\def\lstlistingautorefname{Alg.}

\colorlet{mygreen}{green!60!blue}

\lstdefinestyle{C++}{
    language        = C++,
    frame           = single,
    backgroundcolor = \color{gray!10}, 
    basicstyle      = \ttfamily,
    columns         = fullflexible,
    breaklines      = true,
    keywordstyle    = \color{blue},
    stringstyle     = \color{green},
    commentstyle    = \color{mygreen}\ttfamily
}

\setlength{\parindent}{0.0in}
\setlength{\parskip}{0.05in}

% Edit these as appropriate
\newcommand\course{EC6301}
\newcommand\name{Numerical Opimization}
\newcommand\hwnumber{1}                  % <-- homework number
\newcommand\NetIDa{20211046}           % <-- NetID of person #1
\newcommand\NetIDb{Hyeonjang An}           % <-- NetID of person #2 (Comment this line out for problem sets)

\pagestyle{fancyplain}
\headheight 35pt
\lhead{\NetIDa}
\lhead{\NetIDa\\\NetIDb}                 % <-- Comment this line out for problem sets (make sure you are person #1)
\chead{\textbf{\Large Homework \hwnumber}}
\rhead{\course \\ \name \\ \today}
\lfoot{}
\cfoot{}
\rfoot{\small\thepage}
\headsep 1.5em

\begin{document}

\section*{Problem 1}

Implement the method of bisection , Newtons's, secant, regular falsi.

\begin{enumerate}

\item Optimizing Method Class
\begin{lstlisting}[style=C++]
class Method {
public:
  Method(std::function<float(const float&)> f):function(f){};

  // optimization methods
  float bisection(float start, float end);
  float newtons(float x);
  float secant(float x1, float x0);
  float regular_falsi(float start, float end);

protected:
  // target function as member
  std::function<float(const float&)> function;
};
\end{lstlisting}

\item Bisection method
\lstset{caption={bisection}}
\begin{lstlisting}[style=C++]
float Method::bisection(float start, float end) {
  assert( function(start)*function(end)<0 );

  auto midpoint = (start + end)/2.f;

  if(function(midpoint)==0 || end-start<MIN)
    return midpoint;

  if(function(midpoint)*function(start)<0)
    midpoint = bisection(start, midpoint);
  else
    midpoint = bisection(midpoint, end);

  return midpoint;
}
\end{lstlisting}

\newpage

\item Newtons's method
\lstset{caption={Newton's}}
\begin{lstlisting}[style=C++]
float Method::newtons(float x0) {
  // approximattion of derivative lambda function
  auto d = 
  [](std::function<float(const float&)> func, float x, float eps=1e-6) 
  { 
      return (func(x+eps) - func(x))/eps;
  };

  float x1 = x0;
  while(function(x1)>0.f) {
      float t = x1;
      x1 = t - function(t)/d(function, t);
  }
  return x1;
}
\end{lstlisting}

\item Secant method
\lstset{caption={secant}}
\begin{lstlisting}[style=C++]
// Two point approximation method
float Method::secant(float x1, float x0) {
  // no matter which is bigger
  x1 = std::min(x1, x0);
  x0 = std::max(x1, x0);

  float x2 = MAX; // initial next point
  while(function(x2)>0.f) {
    x2 = x1 - ((x1-x0)/(function(x1)-function(x0))) * function(x1);
    
    x0 = x1; 
    x1 = x2;
  }
  return x2;
}
\end{lstlisting}

\newpage

\item Regular-falsi method
\lstset{caption={regular false}}
\begin{lstlisting}[style=C++]
float Method::regular_falsi(float start, float end) {
  // secant method lambda
  auto sec = [](std::function<float(const float&)> func, float x1, float x0)
  { 
      return x1 - ((x1-x0)/(func(x1)-func(x0))) * func(x1); 
  };

  assert( function(start)*function(end)<0 );

  // new x-axis intersection point
  float x = sec(function, start, end);

  if(function(x)==0 || end-start<MIN)
      return x;

  // do recursivly until the end
  if(function(start)*function(x) < 0)
      x = regular_falsi(start, x);
  else
      x = regular_falsi(x, end);

  return x;
}
\end{lstlisting}

\end{enumerate}

\newpage

\section*{Problem 2}

Discuss thier comparative performance for at least four different problems you generate.

\begin{enumerate}
\item Target functions
\begin{equation*}
  X(m,n) = \left\{\begin{array}{lr}
      x(n), & \text{for } 0\leq n\leq 1\\
      \frac{x(n-1)}{2}, & \text{for } 0\leq n\leq 1\\
      \log_2 \left\lceil n \right\rceil \qquad & \text{for } 0\leq n\leq 1
      \end{array}\right\} = xy
\end{equation*}

\item The derivative of target functions
\begin{equation*}
\end{equation*}

\item Peformance comparision

\end{enumerate}

\end{document}