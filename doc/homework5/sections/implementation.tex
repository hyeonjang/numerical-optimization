% !TEX root=../main.tex
\documentclass{standalone}

\begin{document}
\begin{enumerate}
  \item Implementation
  \begin{enumerate}
  % ##############################################################
  % LinearCG
  % ##############################################################
    \item \textbf{LinearCG}
    \begin{enumerate}
      \item LinearCG is the method for linear function, which can be represented as $\frac{1}{2}x^TAx -bx$
      \item Therefore after deriving matrix $A$ from the function, then we can apply this as input on a computing.
      \item In the implementation, the constructor of LinearCG takes $A$ and $b$ as input argument
      \item As termination condition, the residual $r_k$ should be zero, when the convergence occrus. 
      Because of a numerical calculation, it is not exact zero value, but is set as near zero, threshold $1e-2$
    \end{enumerate}
    
    \begin{itemize}[label=\quad,leftmargin=-5em]
      \item \cppcode[]{../../code/multi/linearcg.hpp}
    \end{itemize}
    \end{enumerate}

  \newpage \begin{enumerate}
  % ##############################################################
  % NonlinearCG
  % ##############################################################
    \item \textbf{NonlinearCG}
    \begin{enumerate}
      \item As similar as LinearCG, to terminate the criterion should be $g_k$ is zero, but as same reason, currently set as $1e-4$
    \end{enumerate}
    
    \begin{itemize}[label=\quad,leftmargin=-5em]
      \item \cppcode[]{../../code/multi/nonlinearcg.hpp}
    \end{itemize}
    \end{enumerate}

  \end{enumerate}
\end{document}