% !TEX root=../main.tex
\documentclass{standalone}

% ==============================================================
% initial point 1.2 1.2 case
% =============================================================
\newcommand{\insertfigures}[5]{
\begin{figure}[ht]
\centering
\begin{multicols}{2}
  \begin{subfigure}{.4\textwidth}
    \includegraphics[width=\linewidth]{#1} \par
  \caption{SteepestDescent}  
  \end{subfigure}
  \begin{subfigure}{.4\textwidth}
    \includegraphics[width=\linewidth]{#2} \par
    \caption{Newtons}
  \end{subfigure}
  \end{multicols} 
  \begin{multicols}{2}
  \begin{subfigure}{.4\textwidth}
    \includegraphics[width=\linewidth]{#3} \par
    \caption{SR1}
  \end{subfigure}
  \begin{subfigure}{.4\textwidth}
    \includegraphics[width=\linewidth]{#4} \par
    \caption{BFGS}
  \end{subfigure}
\end{multicols}
  \caption{#5}
\end{figure}
}

\begin{document}
% ========================================
% Analysis
% =======================================
\begin{enumerate}
\item Discuss for moving the approximate points
\begin{itemize}
  \item I choose the linear function, which always converges, to observe the movement of approximate points.
  \item Excepted the method of steepest descent, we can observe that Newton's method and Quasi-Newton's method takes a large step length.
  \item Therefore, the plottings draw the sharp lines in the kind of Newton's methods
  \item It appears in Figure 1, 4, 5
\end{itemize}
\end{enumerate}
% ======================================== 
% initial point [1.2, 1.2]
% ========================================
\insertfigures
{figures/a SteepestDescent [1.2, 1.2].png}
{figures/a Newtons [1.2, 1.2].png}
{figures/a SR1 [1.2, 1.2].png}
{figures/a BFGS [1.2, 1.2].png}
{$f(x, y)=(x+2y-6)^2 + (2x+y-6)^2$}

\insertfigures
{figures/b SteepestDescent [1.2, 1.2].png}
{figures/b Newtons [1.2, 1.2].png}
{figures/b SR1 [1.2, 1.2].png}
{figures/b BFGS [1.2, 1.2].png}
{$(x, y)=50*(y-x^2)^2 + (1-x)^2$}

\insertfigures
{figures/c SteepestDescent [1.2, 1.2].png}
{figures/c Newtons [1.2, 1.2].png}
{figures/c SR1 [1.2, 1.2].png}
{figures/c BFGS [1.2, 1.2].png}
{$f(x, y)=(1.5-x+xy)^2 + (2.25-x+xy^2)^2 + (2.625 - x+ xy^3)^2$}

% ======================================== 
% Discuss for comparing the approximation
% ========================================
\insertfigures
{figures/a SteepestDescent [5.6,1.2].png}
{figures/a Newtons [5.6,1.2].png}
{figures/a SR1 [5.6,1.2].png}
{figures/a BFGS [5.6,1.2].png}
{Moving comparison[5.6,1.2]: $f(x, y)=(x+2y-6)^2 + (2x+y-6)^2$}

\insertfigures
{figures/a SteepestDescent [-3.5,2.3].png}
{figures/a Newtons [-3.5,2.3].png}
{figures/a SR1 [-3.5,2.3].png}
{figures/a BFGS [-3.5,2.3].png}
{Moving comparison[-3.5,2.3] $f(x, y)=(x+2y-6)^2 + (2x+y-6)^2$}

\insertfigures
{figures/a SteepestDescent [10.5,-8.3].png}
{figures/a Newtons [10.5,-8.3].png}
{figures/a SR1 [10.5,-8.3].png}
{figures/a BFGS [10.5,-8.3].png}
{Moving comparison[10.5,-8.3] $f(x, y)=(x+2y-6)^2 + (2x+y-6)^2$}

\end{document}