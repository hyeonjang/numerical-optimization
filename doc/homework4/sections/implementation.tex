% !TEX root=../main.tex
\documentclass{standalone}

\begin{document}
\begin{enumerate}
    \item Implementation
      \begin{enumerate}
        % ##############################################################
        % Steepest descent
        % ##############################################################
        \item \textbf{Steepest Descent Method}
        \begin{enumerate}
          \item For the steepest descent method, the magnitude of gradient is used as termination criterion
          \item Also, inexact line search method is adapted
        \end{enumerate}
    
        \begin{itemize}[label=\quad,leftmargin=-5em]
          \item \cppcode[]{../../code/multi/cauchys.hpp}
        \end{itemize}
    
        \newpage % ##############################################################
        % Gradient and Hessian
        % ##############################################################
        \item \textbf{Gradient and Hessian}
        \begin{enumerate}
            \item Gradient and Hessian are computed as finite difference method
            \item For accuracy of compututing gradient, 8 values approximation method is taken
            \item It is also same as in the hessian computation 
          \end{enumerate}
      
          \begin{itemize}[label=\quad,leftmargin=-5em]
            \item \cppcode[]{../../code/multivariate.cpp}
          \end{itemize}

        \newpage % ##############################################################
        % Newton's
        % ##############################################################
        \item \textbf{Newton's method}
        \begin{enumerate}
          \item For the steepest descent method, the magnitude of gradient is used as termination criterion
        \end{enumerate}
        \begin{itemize}[label=\quad,leftmargin=-5em]
          \item \cppcode[]{../../code/multi/newtons.hpp}
        \end{itemize}
    
        \newpage % ##############################################################
        % Quasi-Newton's
        % ##############################################################
        \item \textbf{Quasi-Newton's method}
        \begin{enumerate}
          \item Both exact line search (by golden section search method) and inexact line search methods are implemented
          \item I adapt the inexact line search method, because I have thought the overhead is lower than exact line search method
          \item The convergence is depend on finding a step lenght, alpha. 
          In the case of inexact step length, depending on $\rho$ and initial alpha, the step length could be computed badly.
          In that case, the function failed to find a optimal point.
        \end{enumerate}
        \begin{itemize}[label=\quad,leftmargin=-5em]
          \item \cppcode[]{../../code/multi/quasi_newtons.hpp}
        \end{itemize}
    \end{enumerate}
\end{enumerate}
\end{document}