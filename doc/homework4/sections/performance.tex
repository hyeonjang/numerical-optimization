% % !TEX=../main.tex
\documentclass[../main.tex]{subfiles}

\begin{document}
\begin{enumerate}
  \item Convergence speed
  \begin{itemize}
    \item The method of steepest descent has the slowest convergence speed
    \item As before said, I expected the backtracking line search gives more faster speed to search a step length.
    \item But it is not. Because the Quasi-Newton's methods are depending on searching a step length, 
    when the method failed to find an adquate step length, it takes a more time and even fails to converge
    \item In the case of SR1, because of the vulnerability to find a step length, it takes a more time than BFGS
  \end{itemize}
\end{enumerate}
\begin{center}
  \begin{tabular}{| c | c | c | c | c | c | } \hline
  \multirow{2}{*}{initial point} & \multirow{2}{*}{$f(x, y)$}   & \multicolumn{4}{c|}{Performance$(x, y)$} \\ \cline{3-6}
                                  &                              & Steepest descent & Newton's     & SR1         & BFGS \\ \hline
  \multirow{3}{*}{[1.2, 1.2]}     & (a)                          & 151826251312 ns  & 167781 ns    & 4302569 ns  & 1657282 ns  \\ 
                                  & (b)                          & 147027616816 ns  & 200670041 ns & 19256927 ns & 2452604 ns  \\ 
                                  & (c)                          & 270048645501 ns  & fail         & fail        & 20631360 ns \\ \hline
  [5.6,-1.2]                      & (a)                          & 149522732624 ns  & 186664 ns    & 7291947 ns  & 2341505 ns \\ \hline
  [-3.5,2.3]                      & (a)                          & 148679627984 ns  & 203709843 ns & 5417397 ns  & 1443039 ns \\ \hline
  [10.5,-8.3]                     & (a)                          & 131232100735 ns  & 167805 ns    & 5858973 ns  & 3033800 ns \\ \hline
  \end{tabular}
\end{center}


\end{document}