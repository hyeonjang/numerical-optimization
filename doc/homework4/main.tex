\documentclass[12pt,letterpaper]{article}
\usepackage{fullpage}
\usepackage[top=2cm, bottom=4.5cm, left=2.5cm, right=2.5cm]{geometry}
\usepackage{amsmath,amsthm,amsfonts,amssymb,amscd}
\usepackage{lastpage}
\usepackage{enumerate}
\usepackage{fancyhdr}
\usepackage{mathrsfs}
\usepackage{xcolor}
\usepackage{graphicx}
\usepackage{listings}
\usepackage{hyperref}
\usepackage[section]{minted}
\usepackage{hyperref}
\usepackage{multirow}
\usepackage{caption}
\usepackage{float}
\usepackage{subcaption}
\usepackage{multicol}
\usepackage{graphicx}
\usepackage{makecell}
\usepackage{enumitem}
\usepackage{standalone}
\usepackage{import}
\definecolor{mintedbackground}{rgb}{0.95,0.95,0.95}

\hypersetup{%
  colorlinks=true,
  linkcolor=blue,
  linkbordercolor={0 0 1}
}

\newcommand{\insertfigures}[5]{
\begin{figure}[ht]
\centering
\begin{multicols}{2}
  \begin{subfigure}{.4\textwidth}
    \includegraphics[width=\linewidth]{#1} \par
  \caption{SteepestDescent}  
  \end{subfigure}
  \begin{subfigure}{.4\textwidth}
    \includegraphics[width=\linewidth]{#2} \par
    \caption{Newtons}
  \end{subfigure}
  \end{multicols} 
  \begin{multicols}{2}
  \begin{subfigure}{.4\textwidth}
    \includegraphics[width=\linewidth]{#3} \par
    \caption{SR1}
  \end{subfigure}
  \begin{subfigure}{.4\textwidth}
    \includegraphics[width=\linewidth]{#4} \par
    \caption{BFGS}
  \end{subfigure}
\end{multicols}
  \caption{#5}
\end{figure}
}

\renewcommand\lstlistingname{method}
\renewcommand\lstlistlistingname{Algorithms}
\def\lstlistingautorefname{Alg.}

\colorlet{mygreen}{green!60!blue}

\newmintedfile[cppcode]{cpp}{
bgcolor=mintedbackground,
fontfamily=tt,
linenos=true,
numberblanklines=true,
numbersep=5pt,
gobble=0,
frame=leftline,
framerule=0.4pt,
framesep=2mm,
funcnamehighlighting=true,
tabsize=1,
obeytabs=true,
mathescape=false
samepage=true, %with this setting you can force the list to appear on the same page
showspaces=false,
showtabs=false,
texcl=false,
fontsize=\small,
breaklines
}

\setlength{\parindent}{0.0in}
\setlength{\parskip}{0.05in}

% Edit these as appropriate
\newcommand\course{EC6301}
\newcommand\name{Numerical Opimization}
\newcommand\hwnumber{4}                  % <-- homework number
\newcommand\NetIDa{20211046}           % <-- NetID of person #1
\newcommand\NetIDb{Hyeonjang An}           % <-- NetID of person #2 (Comment this line out for problem sets)
\newcommand\github{\url{https://github.com/hyeonjang/numerical-optimization}}
\newcommand\insertcode[1]{{leftskip #1\par}}

\pagestyle{fancyplain}
\headheight 35pt
\lhead{\github\\\NetIDa\\\NetIDb}                 % <-- Comment this line out for problem sets (make sure you are person #1)
\chead{\textbf{\Large Homework \hwnumber}}
\rhead{\course \\ \name \\ \today}
\lfoot{}
\cfoot{}
\rfoot{\small\thepage}
\headsep 1.5em

\begin{document}

\section*{Problem}

\begin{enumerate}
  \item Implement the following numerical methods:
  \begin{enumerate}
    \item The method of steepest descent
    \item Newton's methods
    \item Two Quasi Newton's methods (SR1, BFGS)
  \end{enumerate}
  \item Compare their performance for the following three problems:
  \begin{enumerate}
    \item $f(x, y)=(x+2y-6)^2 + (2x+y-6)^2$
    \item $(x, y)=50*(y-x^2)^2 + (1-x)^2$
    \item $f(x, y)=(1.5-x+xy)^2 + (2.25-x+xy^2)^2 + (2.625 - x+ xy^3)^2$
  \end{enumerate}
  \item First start at (1.2, 1.2) at each function. 
  Then use different starting points to discuss how approximate point is moving on the contour plot of f(x, y)
\end{enumerate}

\section*{Implementation}
\import{sections/}{implementation}

% ##############################################################
% Performance compare
% ##############################################################
\newpage
\section*{Analysis}
\import{sections/}{analysis}

\newpage
\section*{Performance}
\import{sections/}{performance}

% ##############################################################
% Plotting
% ##############################################################
\newpage
\section*{Plotting}
\import{sections/}{plot}

\end{document}